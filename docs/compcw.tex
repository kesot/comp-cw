% !TeX document-id = {fb8a2ef5-cdaf-49da-b79d-0a8152e677cd}
% !TeX TS-program = XeLaTeX
\documentclass[a4paper,12pt]{report}
\raggedbottom
% polyglossia should go first!
\usepackage{polyglossia} % multi-language support

\setdefaultlanguage{russian}
\setmainfont{CMU Serif}
\setsansfont{CMU Sans Serif}
\setmonofont{CMU Typewriter Text}


\setmainlanguage{russian}
\setotherlanguage{english}

\DeclareSymbolFont{letters}{\encodingdefault}{\rmdefault}{m}{it}
\usepackage{amsmath} % math symbols, new environments and stuff
\usepackage{unicode-math} % for changing math font and unicode symbols
\setmathfont{XITS Math}
\allowdisplaybreaks[2]

\usepackage[style=english]{csquotes} % fancy quoting
\usepackage{microtype} % for better font rendering
\usepackage[backend=bibtex, sorting=none]{biblatex} % for bibliography, TODO: replace with biber
\usepackage{hyperref} % for refs and URLs
\usepackage{graphicx} % for images (and title page)
\usepackage{geometry} % for margins in title page
\usepackage{tabu} % for tabulars (and title page)
\usepackage{placeins} % for float barriers
\usepackage{titlesec} % for section break hooks
%\usepackage[justification=centering]{caption} % for forced captions centering
\usepackage{subcaption} % for subfloats
%\usepackage{rotating} % for rotated labels in tables
%\usepackage{tikz} % for TiKZ
%\usepackage{dot2texi} % for inline dot graphs
\usepackage{listings} % for listings 
\usepackage{upquote} % for good-looking quotes in source code (used for custom languages)
\usepackage{multirow} % for multirow cells in tabulars
%\usepackage{afterpage} % for nice landspace floats
%\usepackage{pdflscape} % for landspace orientation
\usepackage{xcolor} % colors!
\usepackage{enumitem} % for unboxed description labels (long ones)
\usepackage{lastpage}
\usepackage{subcaption}
\usepackage{tikz}
\usetikzlibrary{shapes,arrows,patterns}

\lstset{
    frame=none,
    xleftmargin=2pt,
    stepnumber=1,
    numbers=left,
    numbersep=5pt,
    numberstyle=\ttfamily\tiny\color[gray]{0.3},
    belowcaptionskip=\bigskipamount,
    captionpos=b,
    escapeinside={*'}{'*},
    language=haskell,
    tabsize=2,
    emphstyle={\bf},
    commentstyle=\it,
    stringstyle=\mdseries\rmfamily,
    showspaces=false,
    keywordstyle=\bfseries\rmfamily,
    columns=flexible,
    basicstyle=\small\sffamily,
    showstringspaces=false,
    morecomment=[l]\%,
    breaklines=true
}
\renewcommand\lstlistingname{Листинг}

\defaultfontfeatures{Mapping=tex-text} % for converting "--" and "---"

\MakeOuterQuote{"} % enable auto-quotation

\newcommand{\varI}[1]{{\operatorname{\mathit{#1}}}}

% new page and barrier after section, also phantom section after clearpage for
% hyperref to get right page.
% clearpage also outputs all active floats:
\newcommand{\sectionbreak}{\clearpage\phantomsection}
\newcommand{\subsectionbreak}{\FloatBarrier}
\newcommand\numberthis{\addtocounter{equation}{1}\tag{\theequation}}
\renewcommand{\thesection}{\arabic{section}} % no chapters
\numberwithin{equation}{section}
%\usetikzlibrary{shapes,arrows,trees}

\usepackage{expl3}
\ExplSyntaxOn
\cs_new_eq:NN \Repeat \prg_replicate:nn
\ExplSyntaxOff


\DeclareMathOperator{\rank}{rank}
\makeatletter
\newenvironment{sqcases}{%
  \matrix@check\sqcases\env@sqcases
}{%
  \endarray\right.%
}
\def\env@sqcases{%
  \let\@ifnextchar\new@ifnextchar
  \left\lbrack
  \def\arraystretch{1.2}%
  \array{@{}l@{\quad}l@{}}%
}
\makeatother
\setcounter{tocdepth}{3}

%for abstract
%page count \pageref{LastPage}
\usepackage{lastpage}  
%\totalfigures \totaltables
\usepackage[figure,table,xspace]{totalcount}

\usepackage{etoolbox}
\newcounter{totreferences}
\pretocmd{\bibitem}{\addtocounter{totreferences}{1}}{}{}
\makeatletter
     \AtEndDocument{%
       \immediate\write\@mainaux{%
         \string\gdef\string\totref{\number\value{totreferences}}%
       }%
     }
\makeatother

\bibliography{reference_list} 
\begin{document} 

%skip for title page
\setcounter{page}{2}

\tableofcontents

\section{Введение}
В настоящей работе в рамках курса "конструирование компиляторов" реализуется фронтэнд компилятора упрощённой версии языка C--.
В работе приводится использованная грамматика языка, реализуется лексический анализатор, синтаксический анализатор.
Проводится поиск лексических, синтаксических и смысловых ошибок, таких как типовые и другие.
В результате работы создаётся абстрактное синтаксическое дерево, соответствующее входной программе.

\section{Выбор платформы}
Для реализации фронтэнда компилятора решено было использовать язык Haskell.
Был использован дистрибутив MinGHC 7.10.1\cite{minghc}, содержащий компилятор GHC 7.10.1\cite{ghc}, систему сборки и управления пакетами и библиотеками Haskell Cabal 1.22.4.0\cite{cabal} и пакет утилит MSYS\cite{msys}.

Также использовались генератор лексических анализаторов Alex 3.1.4\cite{alex} и генератор обобщённых LR-парсеров Happy 1.18.5\cite{happy}. 
Для сериализации получившегося дерева в формате JSON использовался пакет Aeson 0.6.1.0\cite{aeson}.

Выбор языка был обусловлен его строгой типизацией и удобством отладки благодаря жёстко отслеживаемым побочным эффектам функций, а также наличием большого количества библиотек.

\section{Описание языка}
За грамматику языка была принята упрощённая версия\cite{cmm-grammar} грамматики языка C--, дополненная строковыми литералами.
Грамматика имеет следующий вид:

\begin{align*}
  1.~&~\varI{program}~\rightarrow~\varI{declaration-list} \\
  2.~&~\varI{declaration-list}~\rightarrow~\varI{declaration}~\{~\varI{declaration}~\} \\
  3.~&~\varI{declaration}~\rightarrow~\varI{var-declaration}~|~\varI{fun-declaration} \\
  4.~&~\varI{var-declaration}~\rightarrow~\varI{type-specifier}~\mathbf{ID}~[~\boldsymbol{[}~\mathbf{NUM}~\boldsymbol{]}~]_+~\boldsymbol{;} \\
  5.~&~\varI{type-specifier}~\rightarrow~\mathbf{int}~|~\mathbf{void} \\
  6.~&~\varI{fun-declaration}~\rightarrow~\varI{type-specifier}~\mathbf{ID}~\boldsymbol{(}~\varI{params}~\boldsymbol{)}~\varI{compound-stmt} \\
  7.~&~\varI{params}~\rightarrow~\mathbf{void}~|~\varI{param-list} \\
  8.~&~\varI{param-list}~\rightarrow~\varI{param}~\{~\boldsymbol{,}~\varI{param}~\} \\
  9.~&~\varI{param}~\rightarrow~\varI{type-specifier}~\mathbf{ID}~[~\boldsymbol{[]}~]_+ \\
  10.~&~\varI{compound-stmt}~\rightarrow~\boldsymbol{\{}~\varI{local-declarations}~\varI{statement-list}~\boldsymbol{\}} \\
  11.~&~\varI{local-declarations}~\rightarrow~{~\varI{var-declarations}~} \\
  12.~&~\varI{statement-list}~\rightarrow~{~\varI{statement}~} \\  
  13.~&~\varI{statement}~\rightarrow~\varI{expression-stmt}~ \\
      &~~~~~~~~~~~~~~~~~~~~|~\varI{compound-stmt}~ \\
      &~~~~~~~~~~~~~~~~~~~~|~\varI{selection-stmt}~ \\
      &~~~~~~~~~~~~~~~~~~~~|~\varI{iteration-stmt}~ \\
      &~~~~~~~~~~~~~~~~~~~~|~\varI{assignment-stmt}~ \\
      &~~~~~~~~~~~~~~~~~~~~|~\varI{return-stmt}~ \\
      &~~~~~~~~~~~~~~~~~~~~|~\varI{read-stmt}~ \\
      &~~~~~~~~~~~~~~~~~~~~|~\varI{write-stmt} \\
  14.~&~\varI{expression-stmt}~\rightarrow~\varI{expression}~\boldsymbol{;}~|~\boldsymbol{;} \\
  15.~&~\varI{selection-stmt}~\rightarrow~\mathbf{if}~\boldsymbol{(}~\varI{expression}~\boldsymbol{)}~\varI{statement}~[\mathbf{else}~\varI{statement}]_+ \\
  16.~&~\varI{iteration-stmt}~\rightarrow~\mathbf{while}~\boldsymbol{(}~\varI{expression}~\boldsymbol{)}~\varI{statement} \\
  17.~&~\varI{return-stmt}~\rightarrow~\mathbf{return}~[expression]_+~\boldsymbol{;} \\
  18.~&~\varI{read-stmt}~\rightarrow~\mathbf{read}~\varI{variable}~\boldsymbol{;} \\
  19.~&~\varI{write-stmt}~\rightarrow~\mathbf{write}~\varI{expression}~\boldsymbol{;} \\
  20.~&~\varI{expression}~\rightarrow~\{~\varI{var}~\boldsymbol{=}~\}~\varI{simple-expression} \\
  21.~&~\varI{var}~\rightarrow~\mathbf{ID}~[~\boldsymbol{[}~\varI{expression}~\boldsymbol{]}~]_+ \\
  22.~&~\varI{simple-expression}~\rightarrow~\varI{additive-expression}~[~\varI{relop}~\varI{additive-expression}~]_+ \\
  23.~&~\varI{relop}~\rightarrow~\boldsymbol{<=}~|~\boldsymbol{<}~|~\boldsymbol{>}~|~\boldsymbol{>=}~|~\boldsymbol{==}~|~\boldsymbol{!=} \\
  24.~&~\varI{additive-expression}~\rightarrow~\varI{term}~\{~\varI{addop}~\varI{term}~\} \\
  25.~&~\varI{addop}~\rightarrow~\boldsymbol{+}~|~\boldsymbol{-} \\
  26.~&~\varI{term}~\rightarrow~\varI{factor}~\{~\varI{multop}~\varI{factor}~\} \\
  27.~&~\varI{multop}~\rightarrow~\boldsymbol{*}~|~\boldsymbol{/} \\
  28.~&~\varI{factor}~\rightarrow~\boldsymbol{(}~\varI{expression}~\boldsymbol{)}~|~\mathbf{NUM}~|~\mathbf{ARR}~|~\varI{var}~|~\varI{call} \\
  29.~&~\varI{call}~\rightarrow~\mathbf{ID}~\boldsymbol{(}~\varI{args}~\boldsymbol{)} \\
  30.~&~\varI{args}~\rightarrow~[~\varI{arg-list}~]_+ \\
  31.~&~\varI{arg-list}~\rightarrow~\varI{expression}~\{~\boldsymbol{,}~\varI{expression}~\}
\end{align*}

В грамматике также используются следующие регулярные выражения:
\begin{align*}
  1.~&~\mathbf{ID} = [a-z]+\\
  2.~&~\mathbf{NUM} = [0-9]+ | 'PRINTABLE'\\
  3.~&~\mathbf{ARR} = "PRINTABLE+"\\
  4.~&~\mathbf{PRINTABLE} \text{ -- соответствует любому печатаемому символу}
\end{align*}

\section{Лексический анализ}
\subsection{Входные и выходные структуры данных}
На этапе лексического анализа происходит преобразование разбираемого кода в последовательность токенов, которые в дальнейшем будут обрабатываться в рамках синтаксического анализа и поиска ошибок.

На вход лексического анализатора подаётся строка символов, содержащая всё содержимое файла с исходным кодом.
Выходом лексического анализатора является строка токенов типа, определённого в листинге~\ref{lst:tokentype}.
Выходные токены аннотированы их позицией.
Смысловые значения токенов даны в таблице~\ref{tab:tokens}.

\begin{table}
    \caption{Смысловые значения выходных токенов лексера}
    \label{tab:tokens}
    \begin{tabu}{|l|X[r]|}
    	\hline
    	Токен    & Значение                                      \\ \hline
    	$Symbol$ & Управляющий символ либо ключевое слово        \\ \hline
    	$Num$    & Числовая константа                            \\ \hline
    	$Array$  & Строка, преобразованная к массиву целых чисел \\ \hline
    	$Name$   & имя переменной или функции                    \\ \hline
    \end{tabu}
\end{table}

\begin{lstlisting}[language=haskell,caption={Выходные типы данных лексера},label=lst:tokentype]
data Token =
    Symbol String
    | Array [Int]
    | Num Int
    | Name String 
    deriving (Eq, Show)
data Posed a = Posed (Int,Int) a
\end{lstlisting}

\subsection{Обнаруживаемые ошибки}
На данном этапе единственными ошибками, обнаруживаемыми фронтэндом, вызываются наличием в исходном коде программы символов, неприводимых к описанным выше типам токенов. 
К примеру, лексическую ошибку вызовет встреченный лексером в любом месте программы кроме как внутри одинарных либо двойных кавычек символ $\%$.

Ошибки на данном этапе являются критическими: при возникновении ошибки работа фронтэнда немедленно завершается с выведением сообщения, содержащего номер строки и столбца, где был встречен ошибочный символ.

Пример сообщения об ошибке, вызываемой не входящим в грамматику языка символом, приведён в листинге~\ref{lst:error-alex}.

\begin{lstlisting}[float={},language=bash,caption={Сообщение о лексической ошибке},label=lst:error-alex]
lexical error at line 39, column 15
\end{lstlisting}

\subsection{Реализация лексера}
Лексер был реализован с помощью средства генерации лексеров Alex.

Конфигурационный файл Alex содержит регулярные выражения, соответствующие различным токенам, набор правил для их преобразования и описание выходных структур данных.
Также в нём содержатся вспомогательные функции, используемые в правилах преобразования токенов.
Всё содержимое конфигурационного файла приведено в листинге в приложении 1.
Haskel-модуль генерируется из конфигурационного файла с помощью утилиты командной строки \textit{alex}.

\section{Синтаксический анализ}
\subsection{Входные и выходные структуры данных}
На этапе синтаксического анализа происходит преобразование последовательности токенов, полученных лексером, в дерево выражения, которое будет в дальнейшем проверяться на типовую согласованность и отсутствие неопределённых имён.

На вход синтаксического анализатора подаётся список аннотированных позицией токенов, полученных лексером из исходного кода.
Выходом анализатора является преобразованное дерево разбора исходной грамматики.
От собственно дерева разбора оно отличается отсутствием элементов, не являющихся необходимыми для дальнейших расчётов -- так, в выходном дереве нет отдельных узлов, соответствующих нетерминалам \textit{factor} и \textit{term}, и они оба соответствуют узлу \textit{Expression} в результирующем дереве.

Определение выходных типов парсера приведено в листинге~\ref{lst:parsertype}.

\begin{lstlisting}[language=haskell,caption={Выходные типы данных парсера},label=lst:parsertype]
type Reference = (Posed String, Maybe (TExpression))
data TExpression =
    ComplEx [Reference] TExpression
    | CallEx (Posed String) [TExpression]
    | Retrieval Reference
    | StringLiteral (Posed [Int])
    | NumLiteral (Posed Int)
    deriving (Show, Eq)
data TStatement =
    CompSta [TDeclaration] [TStatement]
    | SelSta TExpression TStatement (Maybe TStatement)
    | IterSta TExpression TStatement
    | RetSta (Int,Int) (Maybe TExpression)
    | ReadSta Reference
    | ExpSta TExpression
    | EmpSta
    deriving (Show, Eq)
data TDeclaration =
    Intdecl (Posed String)
    | Arrdecl (Posed String) (Posed Int)
    | Fundecl (Posed String) [TDeclaration] TStatement
    | Procdecl (Posed String) [TDeclaration] TStatement
    deriving (Show, Eq)
\end{lstlisting}

\subsection{Обнаруживаемые ошибки}
На данном этапе обнаруживаются ошибки несоответствия исходного кода грамматике языка.
Ошибки данного типа также критические: хотя более сложные парсеры и могут восстанавливаться после ошибок, к примеру, с помощью поиска синхронизующих элементов~\cite{wikiparseerror}, но данный механизм является весьма громоздким в реализации.

При возникновении ошибки работа фронтэнда немедленно завершается с выведением сообщения, содержащего номер строки и столбца, где был встречен ошибочный токен.

Пример сообщения об ошибке, вызываемой ошибочным фрагментом кода \textit{"int void arrln;"}, приведён в листинге~\ref{lst:error-happy}.

\begin{lstlisting}[language=bash,caption={Сообщение о синтаксической ошибке},label=lst:error-happy]
Syntax error at (39,9): unexpected symbol "Symbol "void""
\end{lstlisting}

\subsection{Реализация парсера}
Парсер был реализован с помощью средства генерации GLR-парсеров Happy.

Конфигурационный файл Happy содержит:
\begin{enumerate}
  \item описание выходных типов данных в языке Haskell
  \item функцию-обработчик ошибок разбора
  \item определение терминалов и соответствующих им токенов
  \item информацию о ассоциативности различных терминалов
  \item правила вывода, содержащие цепочки из элементов объединённого алфавита
  \item правила преобразования правых частей правил к соответствующим левым частям структурам данных
\end{enumerate}

Haskell-модуль генерируется из конфигурационного файла с помощью утилиты командной строки \textit{happy}.
Содержимое конфигурационного файла Happy приведено в приложении 2.

\section{Поиск ошибок}


\section{Список литературы}
\printbibliography[heading=none]


\section*{\titleline[r]{Приложение 1}} \addcontentsline{toc}{section}{Приложение 1}
\subsection*{Конфигурационный файл Alex}
\lstinputlisting{../src/cmm_alex.x}
\clearpage

\section*{\titleline[r]{Приложение 2}} \addcontentsline{toc}{section}{Приложение 2}
\subsection*{Конфигурационный файл Happy}
\lstinputlisting{../src/cmm_happy.y}
\clearpage

\section*{\titleline[r]{Приложение 3}} \addcontentsline{toc}{section}{Приложение 3}
\subsection*{Модуль проверки логических ошибок}
\lstinputlisting{../src/checker.hs}
\clearpage

\section*{\titleline[r]{Приложение 4}} \addcontentsline{toc}{section}{Приложение 4}
\subsection*{Модуль построения выходного синтаксического дерева}
\lstinputlisting{../src/astbuilder.hs}
\clearpage

\section*{\titleline[r]{Приложение 5}} \addcontentsline{toc}{section}{Приложение 5}
\subsection*{Главный модуль программы}
\lstinputlisting{../src/test.hs}
\clearpage

\section*{\titleline[r]{Приложение 6}} \addcontentsline{toc}{section}{Приложение 6}
\subsection*{Модуль вспомогательных утилит}
\lstinputlisting{../src/dictutils.hs}
\clearpage

\end{document}